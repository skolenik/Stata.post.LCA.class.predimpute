\inserttype[st0001]{article}
\author{Kolenikov}{%
  Stas Kolenikov\\NORC\\Columbia, Missouri, USA\\kolenikov-stas@norc.org
}
\title[Post-estimation for LCA via MI]{Inference for imputed latent classes using multiple imputation}
\maketitle

\begin{abstract}
This is an example article.  You should change the \verb+\input{}+ line in
\texttt{main.tex} to point to your file.  If this is your first submission to
the {\sl Stata Journal}, please read the following ``getting started''
information.

\keywords{\inserttag, postlca\_class\_predpute, latent class analysis, multiple imputation}
\end{abstract}

\section{Latent class analysis}

Latent class analysis 

Running example

\begin{stlog}
. webuse gsem_lca1
\smallskip
. gsem (accident play insurance stock <- ), logit lclass(C 2)
\end{stlog}

Researchers are often interested in describing the latent classes
or using these classes in analysis as predictors or as moderators.
The official \semref{gsem postestimation} commands provide
limited possibilities, namely reporting of the means
of the dependent variables by class via \stcmd{estat lcmean}.
For nearly all meaningful applications of LCA, this is insufficient.

One possible approach is to predict the modal class for each
observation, and use it in subsequent downstream analyses 
treating that as fixed:



The program distributed with the current package,
\stcmd{postlca\_class\_predpute}, provides a pathway for the appropriate
statistical inference that would account for uncertainty in class prediction.
This is achieved through the mechanics of multiple imputation
\citep{vanbuuren:2018:fimd2}. 
The name is supposed to convey that
\begin{enumerate}
  \item it is supposed to be run after LCA as a post-estimation command;
  \item it predicts / imputes the latent classes.
\end{enumerate}

\section{The new command}

Imputation of latent classes, a \stcmd{gsem} postestimation command:

\begin{stverbatim}
\begin{verbatim}
\begin{stsyntax}
    postlca\_class\_predpute,
    lcimpute(\varname)
    addm(\num)
    \optional{ seed(\num) }
\end{stsyntax}
\end{verbatim}
\end{stverbatim}

\hangpara{
  \stcmd{lcimpute(}\varname\stcmd{)} 
  specifies the name of the latent class variable to be imputed.
  This option is required.
}

\hangpara{
  \stcmd{addm(}\num{)} specifies the number of imputations to be created.
  This option is required.
}

\hangpara{
  \stcmd{seed(}\num\stcmd{)} specifies the random number seed.
}

\section{Examples}

\subsection{Stata manual data set example}

The LCA capabilities of Stata are exemplified in [SEM] \textbf{Example 50g}:

\begin{stlog}
. frame change default
{\smallskip}
. cap frame gsem_lca1: clear
{\smallskip}
. cap frame drop gsem_lca1 
{\smallskip}
. frame create gsem_lca1
{\smallskip}
. frame change gsem_lca1
{\smallskip}
. 
. webuse gsem_lca1.dta, clear
(Latent class analysis)
{\smallskip}
. describe
{\smallskip}
Contains data from https://www.stata-press.com/data/r18/gsem_lca1.dta
 Observations:           216                  Latent class analysis
    Variables:             4                  17 Jan 2023 12:52
                                              (_dta has notes)
\HLI{95}
Variable      Storage   Display    Value
    name         type    format    label      Variable label
\HLI{95}
accident        byte    \%9.0g                 Would testify against friend in accident case
play            byte    \%9.0g                 Would give negative review of friend's play
insurance       byte    \%9.0g                 Would disclose health concerns to friend's
                                                insurance company
stock           byte    \%9.0g                 Would keep company secret from friend
\HLI{95}
Sorted by: accident  play  insurance  stock
{\smallskip}
. gsem (accident play insurance stock <-), logit lclass(C 2)
{\oom}
Generalized structural equation model                      Number of obs = 216
Log likelihood = -504.46767
{\smallskip}
\HLI{13}{\TOPT}\HLI{64}
             {\VBAR} Coefficient  Std. err.      z    P>|z|     [95\% conf. interval]
\HLI{13}{\PLUS}\HLI{64}
1.C          {\VBAR}  (base outcome)
\HLI{13}{\PLUS}\HLI{64}
2.C          {\VBAR}
       _cons {\VBAR}  -.9482041   .2886333    -3.29   0.001    -1.513915   -.3824933
\HLI{13}{\BOTT}\HLI{64}
{\smallskip}
Class:    1        
{\smallskip}
Response: accident 
Family:   Bernoulli
Link:     Logit    
{\smallskip}
Response: play     
Family:   Bernoulli
Link:     Logit    
{\smallskip}
Response: insurance
Family:   Bernoulli
Link:     Logit    
{\smallskip}
Response: stock    
Family:   Bernoulli
Link:     Logit    
{\smallskip}
\HLI{13}{\TOPT}\HLI{64}
             {\VBAR} Coefficient  Std. err.      z    P>|z|     [95\% conf. interval]
\HLI{13}{\PLUS}\HLI{64}
accident     {\VBAR}
       _cons {\VBAR}   .9128742   .1974695     4.62   0.000     .5258411    1.299907
\HLI{13}{\PLUS}\HLI{64}
play         {\VBAR}
       _cons {\VBAR}  -.7099072   .2249096    -3.16   0.002    -1.150722   -.2690926
\HLI{13}{\PLUS}\HLI{64}
insurance    {\VBAR}
       _cons {\VBAR}  -.6014307   .2123096    -2.83   0.005     -1.01755   -.1853115
\HLI{13}{\PLUS}\HLI{64}
stock        {\VBAR}
       _cons {\VBAR}  -1.880142   .3337665    -5.63   0.000    -2.534312   -1.225972
\HLI{13}{\BOTT}\HLI{64}
{\smallskip}
Class:    2        
{\oom}
\HLI{13}{\TOPT}\HLI{64}
             {\VBAR} Coefficient  Std. err.      z    P>|z|     [95\% conf. interval]
\HLI{13}{\PLUS}\HLI{64}
accident     {\VBAR}
       _cons {\VBAR}   4.983017   3.745987     1.33   0.183    -2.358982    12.32502
\HLI{13}{\PLUS}\HLI{64}
play         {\VBAR}
       _cons {\VBAR}   2.747366   1.165853     2.36   0.018     .4623372    5.032395
\HLI{13}{\PLUS}\HLI{64}
insurance    {\VBAR}
       _cons {\VBAR}   2.534582   .9644841     2.63   0.009     .6442279    4.424936
\HLI{13}{\PLUS}\HLI{64}
stock        {\VBAR}
       _cons {\VBAR}   1.203416   .5361735     2.24   0.025     .1525356    2.254297
\HLI{13}{\BOTT}\HLI{64}
\nullskip
\end{stlog}

One of the official post-estimation commands available after
\stcmd{gsem, lclass()} is the computation of the class-specific means
of the outcome variables:

\begin{stlog}
. set rmsg on
r; t=0.00 14:45:54
{\smallskip}
. estat lcprob
{\smallskip}
Latent class marginal probabilities                        Number of obs = 216
{\smallskip}
\HLI{13}{\TOPT}\HLI{48}
             {\VBAR}            Delta-method
             {\VBAR}     Margin   std. err.     [95\% conf. interval]
\HLI{13}{\PLUS}\HLI{48}
           C {\VBAR}
          1  {\VBAR}   .7207539   .0580926      .5944743    .8196407
          2  {\VBAR}   .2792461   .0580926      .1803593    .4055257
\HLI{13}{\BOTT}\HLI{48}
r; t=1.13 14:45:55
{\smallskip}
. estat lcmean
{\smallskip}
Latent class marginal means                                Number of obs = 216
{\smallskip}
\HLI{13}{\TOPT}\HLI{48}
             {\VBAR}            Delta-method
             {\VBAR}     Margin   std. err.     [95\% conf. interval]
\HLI{13}{\PLUS}\HLI{48}
1            {\VBAR}
    accident {\VBAR}   .7135879   .0403588      .6285126    .7858194
        play {\VBAR}   .3296193   .0496984      .2403573    .4331299
   insurance {\VBAR}   .3540164   .0485528      .2655049    .4538042
       stock {\VBAR}   .1323726   .0383331      .0734875    .2268872
\HLI{13}{\PLUS}\HLI{48}
2            {\VBAR}
    accident {\VBAR}   .9931933   .0253243      .0863544    .9999956
        play {\VBAR}   .9397644   .0659957      .6135685    .9935191
   insurance {\VBAR}   .9265309   .0656538      .6557086    .9881667
       stock {\VBAR}    .769132   .0952072      .5380601    .9050206
\HLI{13}{\BOTT}\HLI{48}
r; t=4.48 14:46:00
{\smallskip}
. set rmsg off
{\smallskip}
\nullskip
\end{stlog}

The mutiple imputation version of this estimation task could
look as follows:

\begin{stlog}
. set rmsg on
r; t=0.00 14:46:00
{\smallskip}
. postlca_class_predpute, lcimpute(lclass) addm(10) seed(12345)
(216 missing values generated)
(10 imputations added; {\sltt{M}} = 10)
r; t=0.05 14:46:00
{\smallskip}
. mi estimate : prop lclass
{\smallskip}
Multiple-imputation estimates     Imputations     =         10
Proportion estimation             Number of obs   =        216
                                  Average RVI     =     0.4594
                                  Largest FMI     =     0.3319
                                  Complete DF     =        215
DF adjustment:   Small sample     DF:     min     =      55.99
                                          avg     =      55.99
Within VCE type:     Analytic             max     =      55.99
{\smallskip}
\HLI{13}{\TOPT}\HLI{48}
             {\VBAR}                                   Normal
             {\VBAR} Proportion   Std. err.     [95\% conf. interval]
\HLI{13}{\PLUS}\HLI{48}
      lclass {\VBAR}
          1  {\VBAR}   .7236111   .0367281      .6500355    .7971867
          2  {\VBAR}   .2763889   .0367281      .2028133    .3499645
\HLI{13}{\BOTT}\HLI{48}
r; t=2.01 14:46:02
{\smallskip}
. mi estimate : mean accident, over(lclass)
{\smallskip}
Multiple-imputation estimates     Imputations     =         10
Mean estimation                   Number of obs   =        216
                                  Average RVI     =     0.3882
                                  Largest FMI     =     0.4485
                                  Complete DF     =        215
DF adjustment:   Small sample     DF:     min     =      35.59
                                          avg     =     116.62
Within VCE type:     Analytic             max     =     197.64
{\smallskip}
\HLI{18}{\TOPT}\HLI{48}
                  {\VBAR}       Mean   Std. err.     [95\% conf. interval]
\HLI{18}{\PLUS}\HLI{48}
c.accident@lclass {\VBAR}
               1  {\VBAR}   .7144964   .0369935      .6415438    .7874491
               2  {\VBAR}   .9934973   .0135709      .9659633    1.021031
\HLI{18}{\BOTT}\HLI{48}
Note: Numbers of observations in {\bftt{e(_N)}} vary among imputations.
r; t=2.40 14:46:04
{\smallskip}
. set rmsg off
{\smallskip}
\nullskip
\end{stlog}

The name of the latent class variable (here, \stcmd{lclass})
and the number of imputations are required. The seed is optional,
but of course is strongly recommended for reproducibility of the results,
as the underlying data are randomly simulated.
The multiple imputation version is notably faster.

\subsection{NHANES complex survey data}

In many important and realistic applications of LCA, including the case
that necessitated the development of this package, the data come from
complex survey designs that require setting the data up for the appropriate
survey-design adjusted analyses. See \svyref{svyset}, \miref{mi svyset},
and \citet{kolenikov:pitblado:2014}.

As one of many diagnostic outputs of MI, the increase in variances / standard errors
due to imputations serves as an indication of how much of a problem
would treating the singly imputed (e.g. modal probability) latent classes would have been



\newpage

% discussion of the Stata Journal document class.
\input sj.tex
% discussion of the Stata Press LaTeX package for Stata output.
\input stata.tex

\bibliographystyle{sj}
\bibliography{sj}

\begin{aboutauthors}
Stas Kolenikov is Principal Statistician at NORC who has been
using Stata and writing Stata programs for about 25 years.
He had worked on economic welfare and inequality, spatiotemporal
environmental statistics, mixture models, missing data,
multiple imputation, structural equations with latent variables,
resampling methods, complex sampling designs, survey weights,
Bayesian mixed models, combining probability and non-probability samples,
latent class analysis, and likely some other stuff, too.
\end{aboutauthors}

\endinput
