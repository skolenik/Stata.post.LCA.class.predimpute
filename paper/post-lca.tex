% readme.tex -- a short example of how each Stata Journal insert should be
% organized.

\inserttype[st0001]{article}
\author{Kolenikov}{%
  Stas Kolenikov\\NORC\\Columbia, Missouri, USA\\kolenikov-stas@norc.org
}
\title[Post-estimation for LCA via MI]{Inference for imputed latent classes using multiple imputation}
\maketitle

\begin{abstract}
This is an example article.  You should change the \verb+\input{}+ line in
\texttt{main.tex} to point to your file.  If this is your first submission to
the {\sl Stata Journal}, please read the following ``getting started''
information.

\keywords{\inserttag, postlca\_class\_predpute, latent class analysis, multiple imputation}
\end{abstract}

\section{Latent class analysis}

Researchers are often interested in describing the latent classes
or using these classes in analysis as predictors or as moderators.


\section{New command}

\section{Examples}



\newpage

% discussion of the Stata Journal document class.
\input sj.tex
% discussion of the Stata Press LaTeX package for Stata output.
\input stata.tex

\bibliographystyle{sj}
\bibliography{sj}

\begin{aboutauthors}
Stas Kolenikova is Principal Statistician at NORC.
\end{aboutauthors}

\endinput
