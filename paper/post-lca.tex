\inserttype[st0001]{article}
\author{Kolenikov}{%
  Stas Kolenikov\\NORC\\Columbia, Missouri, USA\\kolenikov-stas@norc.org
}
\title[Post-estimation for LCA via MI]{Inference for imputed latent classes using multiple imputation}
\maketitle

\begin{abstract}
I introduce a command to multiply impute latent classes
following \stcmd{gsem, lclass()} latent class analysis. 
This allows properly propagating uncertainty in class 
membership to downstream analysis that may characterize
the demographic composition of the classes, or use 
the class as a predictor variable in statitsical models.

\keywords{\inserttag, postlca\_class\_predpute, latent class analysis, multiple imputation}
\end{abstract}

\section{Latent class analysis}

Latent class analysis (LCA) is a commonly used statistical and quantitative
social science technique of modeling counts in high dimensional contingency tables,
or tables of associations of categorical variables
\citet{hagenaars:mccutcheon:2002,mccutcheon:1987}. 
LCA is a form of loglinear modeling, so let us explain that first.
If the researcher has several categorical variables $X_1, X_2, \ldots, X_p$
with categories 1 through $m_j, j=1, \ldots, p$,
at their disposal, and can produce counts $n_{{k_1}{k_2}\ldots{k_p}}$ 
in a complete $p$-dimensional table, the first step could be modeling
in main effects:

\begin{equation}
  \label{eq:loglinear:main}
  \Expect \log n_{{k_1}{k_2}\ldots{k_p}} = 
  \mathrm{offset} + 
  \sum_{j=1}^p \sum_{k=1}^{m_j} \beta_{jk_j} 
\end{equation}

\noindent
with applicable identification constraints (such as the sum of the coefficients
of a single variable is zero, or the coefficient for the first category 
of a variable is zero). Parameter estimates can be obtained by maximum
likelihood, as equation (\ref{eq:loglinear:main}) is a Poisson regression model.
This model can be denoted as $X_1 + X_2 + \ldots + X_p$ main effects model.
The fit of the model is assessed by the Pearson $\chi^2$ test comparing the
expected vs. observed cell counts, or the likelihood ratio test against a saturated
model where each cell has its own coefficient. If the model were to be found inadequate,
the researcher can entertain adding interactions, e.g. the interaction of 
$X_1$ and $X_2$ would have $m_1 \times m_2$ terms for each pair of values 
of these variables, rather than $m_1 + m_2$ main effects:

\begin{equation}
  \label{eq:loglinear:x1x2}
  \Expect \log n_{{k_1}{k_2}\ldots{k_p}} = 
  \mathrm{offset} + 
  \sum_{k_1=1}^{m_1} \sum_{k_2=1}^{m_2} \beta_{12,k_1k_2} +
  \sum_{j=3}^p \sum_{k=1}^{m_j} \beta_{jk_j} 
\end{equation}

This model can be denoted as $X_1 \# X_2 + X_3 + \ldots + X_p$.

In the loglinear model notation, the latent class models 
are models of the form $C \# (X_1 + X_2 + \ldots + X_p)$.
Categorical latent variable $C$ is the latent class.
The model is now a mixture of Poisson regressions,
and maximum likelihood estimation additionally involves 
estimating the prevalence of each class of $C$.

Further extensions of latent class analysis may include:

\begin{enumerate}
  \item Analysis with interactions of the observed variables;
  \item Analysis with complex survey data (in which case estimation proceeds
      with \stcmd{svy} prefix, and the counts are the weighted estimates 
      of the population totals in cells);
  \item Constrained analyses with structural zeroes or ones 
      (e.g. that every member of class $C=1$ must have the value
      $X_1=1$);
  \item Constrained analyses where some variables have identical
      coefficients across classes.
\end{enumerate}

\subsection{Official Stata implementation}

Official Stata \stcmd{gsem, lclass()} implements 
the \textit{main effects} LCA. The syntax is that of the SEM
families, with the variables that the arrow points to
interpreted as the outcome variables, and the latent class
variable being the source of the arrow:

\begin{stlog}
. webuse gsem_lca1
\smallskip
. gsem (accident play insurance stock <- ), logit lclass(C 2)
\end{stlog}

The goodness of fit test against the free tabulation counts
is provided by \stcmd{estat gof} (not available after
the complex survey data analysis.)

As LCA is implemented through \stcmd{gsem},
all the link functions and generalized linear model families
are supported, extending the ``mainstream'' LCA.

\subsection{Examples }

LCA has found use in analyses of complicated economic concepts
from survey data, and in assessment of measurement errors
that arise in the process.

\citet{biemer:2004} provided analysis of labor force classification
status (employed, unemployed, and out of labor force) following changes
in the survey instrument used in the Current Population Survey (CPS).
The latent classes are the true 
LFS categories, and the observed variables are the corresponding 
survey measurements, demographics, and survey interview mode.
The model is a variation of LCA that accounts for the survey methodology
aspects of CPS: 
its panel nature (responses are collected over four consecutive months) 
and response mode (proxy reporting when a family member provides
the survey responses rather than the target person.)
He found that measurement of being employed and not being in labor force
are highly accurate (98\% and 97\% accuracy) while measurement of 
being unemployed is much less accurate (between 74\% and 81\% depending
on the analysis year.) LCA allowed to further attribute the drop in
accuracy of the unemployment status measurement to proxy reporting,
and to the problems with measuring the employment status when 
the worker is laid off.

\citet{kolenikov:daley:2017} analyzed the latent classes
of employees 
using the U.S. Department of Labor Worker Classification Survey.
The observed variables were (composite) 
self-report of the employment
status (are you an employee at your job; do you refer to your work
as your business, your client, your job, etc.); 
tax status
(the forms that the worker receives from their firm: W-2, 1099, K-1, etc.);
behavioral control 
(functions the worker performs and the degree of control over these functions,
such as direct reporting to somebody, schedule, permission to leave, etc.);
and non-control composite
(hired for fixed time or specific project). They found the best fitting
model to contain three classes: employees-and-they-know-it (59\%),
nonemployees-and-they-know-it (24\%), and confused (17\%) who
classify themselves as employees but their tax documentation
is unclear, and other variables tend to place them into non-employee status.

\subsection{Scope for this package}

Researchers are often interested in describing the latent classes
or using these classes in analysis as predictors or as moderators.
The official \semref{gsem postestimation} commands provide
limited possibilities, namely reporting of the means
of the dependent variables by class via \stcmd{estat lcmean}.
For nearly all meaningful applications of LCA, this is insufficient.

The program distributed with the current package,
\stcmd{postlca\_class\_predpute}, provides a pathway for the appropriate
statistical inference that would account for uncertainty in class prediction.
This is achieved through the mechanics of multiple imputation
\citep{vanbuuren:2018:fimd2}. 
The name is supposed to convey that
\begin{enumerate}
  \item it is supposed to be run after LCA as a post-estimation command;
  \item it predicts / imputes the latent classes.
\end{enumerate}

\section{The new command}

Imputation of latent classes, a \stcmd{gsem} postestimation command:

\begin{stverbatim}
\begin{verbatim}
\begin{stsyntax}
    postlca\_class\_predpute,
    lcimpute(\varname)
    addm(\num)
    \optional{ seed(\num) }
\end{stsyntax}
\end{verbatim}
\end{stverbatim}

\hangpara{
  \stcmd{lcimpute(}\varname\stcmd{)} 
  specifies the name of the latent class variable to be imputed.
  This option is required.
}

\hangpara{
  \stcmd{addm(}\num{)} specifies the number of imputations to be created.
  This option is required.
}

\hangpara{
  \stcmd{seed(}\num\stcmd{)} specifies the random number seed.
}

\section{Examples}

\subsection{Stata manual data set example}

The LCA capabilities of Stata are exemplified in [SEM] \textbf{Example 50g}:

\begin{stlog}
. frame change default
{\smallskip}
. cap frame gsem_lca1: clear
{\smallskip}
. cap frame drop gsem_lca1 
{\smallskip}
. frame create gsem_lca1
{\smallskip}
. frame change gsem_lca1
{\smallskip}
. 
. webuse gsem_lca1.dta, clear
(Latent class analysis)
{\smallskip}
. describe
{\smallskip}
Contains data from https://www.stata-press.com/data/r18/gsem_lca1.dta
 Observations:           216                  Latent class analysis
    Variables:             4                  17 Jan 2023 12:52
                                              (_dta has notes)
\HLI{95}
Variable      Storage   Display    Value
    name         type    format    label      Variable label
\HLI{95}
accident        byte    \%9.0g                 Would testify against friend in accident case
play            byte    \%9.0g                 Would give negative review of friend's play
insurance       byte    \%9.0g                 Would disclose health concerns to friend's
                                                insurance company
stock           byte    \%9.0g                 Would keep company secret from friend
\HLI{95}
Sorted by: accident  play  insurance  stock
{\smallskip}
. gsem (accident play insurance stock <-), logit lclass(C 2)
{\oom}
Generalized structural equation model                      Number of obs = 216
Log likelihood = -504.46767
{\smallskip}
\HLI{13}{\TOPT}\HLI{64}
             {\VBAR} Coefficient  Std. err.      z    P>|z|     [95\% conf. interval]
\HLI{13}{\PLUS}\HLI{64}
1.C          {\VBAR}  (base outcome)
\HLI{13}{\PLUS}\HLI{64}
2.C          {\VBAR}
       _cons {\VBAR}  -.9482041   .2886333    -3.29   0.001    -1.513915   -.3824933
\HLI{13}{\BOTT}\HLI{64}
{\smallskip}
Class:    1        
{\smallskip}
Response: accident 
Family:   Bernoulli
Link:     Logit    
{\smallskip}
Response: play     
Family:   Bernoulli
Link:     Logit    
{\smallskip}
Response: insurance
Family:   Bernoulli
Link:     Logit    
{\smallskip}
Response: stock    
Family:   Bernoulli
Link:     Logit    
{\smallskip}
\HLI{13}{\TOPT}\HLI{64}
             {\VBAR} Coefficient  Std. err.      z    P>|z|     [95\% conf. interval]
\HLI{13}{\PLUS}\HLI{64}
accident     {\VBAR}
       _cons {\VBAR}   .9128742   .1974695     4.62   0.000     .5258411    1.299907
\HLI{13}{\PLUS}\HLI{64}
play         {\VBAR}
       _cons {\VBAR}  -.7099072   .2249096    -3.16   0.002    -1.150722   -.2690926
\HLI{13}{\PLUS}\HLI{64}
insurance    {\VBAR}
       _cons {\VBAR}  -.6014307   .2123096    -2.83   0.005     -1.01755   -.1853115
\HLI{13}{\PLUS}\HLI{64}
stock        {\VBAR}
       _cons {\VBAR}  -1.880142   .3337665    -5.63   0.000    -2.534312   -1.225972
\HLI{13}{\BOTT}\HLI{64}
{\smallskip}
Class:    2        
{\oom}
\HLI{13}{\TOPT}\HLI{64}
             {\VBAR} Coefficient  Std. err.      z    P>|z|     [95\% conf. interval]
\HLI{13}{\PLUS}\HLI{64}
accident     {\VBAR}
       _cons {\VBAR}   4.983017   3.745987     1.33   0.183    -2.358982    12.32502
\HLI{13}{\PLUS}\HLI{64}
play         {\VBAR}
       _cons {\VBAR}   2.747366   1.165853     2.36   0.018     .4623372    5.032395
\HLI{13}{\PLUS}\HLI{64}
insurance    {\VBAR}
       _cons {\VBAR}   2.534582   .9644841     2.63   0.009     .6442279    4.424936
\HLI{13}{\PLUS}\HLI{64}
stock        {\VBAR}
       _cons {\VBAR}   1.203416   .5361735     2.24   0.025     .1525356    2.254297
\HLI{13}{\BOTT}\HLI{64}
\nullskip
\end{stlog}

One of the official post-estimation commands available after
\stcmd{gsem, lclass()} is the computation of the class-specific means
of the outcome variables:

\begin{stlog}
. set rmsg on
r; t=0.00 14:45:54
{\smallskip}
. estat lcprob
{\smallskip}
Latent class marginal probabilities                        Number of obs = 216
{\smallskip}
\HLI{13}{\TOPT}\HLI{48}
             {\VBAR}            Delta-method
             {\VBAR}     Margin   std. err.     [95\% conf. interval]
\HLI{13}{\PLUS}\HLI{48}
           C {\VBAR}
          1  {\VBAR}   .7207539   .0580926      .5944743    .8196407
          2  {\VBAR}   .2792461   .0580926      .1803593    .4055257
\HLI{13}{\BOTT}\HLI{48}
r; t=1.13 14:45:55
{\smallskip}
. estat lcmean
{\smallskip}
Latent class marginal means                                Number of obs = 216
{\smallskip}
\HLI{13}{\TOPT}\HLI{48}
             {\VBAR}            Delta-method
             {\VBAR}     Margin   std. err.     [95\% conf. interval]
\HLI{13}{\PLUS}\HLI{48}
1            {\VBAR}
    accident {\VBAR}   .7135879   .0403588      .6285126    .7858194
        play {\VBAR}   .3296193   .0496984      .2403573    .4331299
   insurance {\VBAR}   .3540164   .0485528      .2655049    .4538042
       stock {\VBAR}   .1323726   .0383331      .0734875    .2268872
\HLI{13}{\PLUS}\HLI{48}
2            {\VBAR}
    accident {\VBAR}   .9931933   .0253243      .0863544    .9999956
        play {\VBAR}   .9397644   .0659957      .6135685    .9935191
   insurance {\VBAR}   .9265309   .0656538      .6557086    .9881667
       stock {\VBAR}    .769132   .0952072      .5380601    .9050206
\HLI{13}{\BOTT}\HLI{48}
r; t=4.48 14:46:00
{\smallskip}
. set rmsg off
{\smallskip}
\nullskip
\end{stlog}

The mutiple imputation version of this estimation task could
look as follows:

\begin{stlog}
. set rmsg on
r; t=0.00 14:46:00
{\smallskip}
. postlca_class_predpute, lcimpute(lclass) addm(10) seed(12345)
(216 missing values generated)
(10 imputations added; {\sltt{M}} = 10)
r; t=0.05 14:46:00
{\smallskip}
. mi estimate : prop lclass
{\smallskip}
Multiple-imputation estimates     Imputations     =         10
Proportion estimation             Number of obs   =        216
                                  Average RVI     =     0.4594
                                  Largest FMI     =     0.3319
                                  Complete DF     =        215
DF adjustment:   Small sample     DF:     min     =      55.99
                                          avg     =      55.99
Within VCE type:     Analytic             max     =      55.99
{\smallskip}
\HLI{13}{\TOPT}\HLI{48}
             {\VBAR}                                   Normal
             {\VBAR} Proportion   Std. err.     [95\% conf. interval]
\HLI{13}{\PLUS}\HLI{48}
      lclass {\VBAR}
          1  {\VBAR}   .7236111   .0367281      .6500355    .7971867
          2  {\VBAR}   .2763889   .0367281      .2028133    .3499645
\HLI{13}{\BOTT}\HLI{48}
r; t=2.01 14:46:02
{\smallskip}
. mi estimate : mean accident, over(lclass)
{\smallskip}
Multiple-imputation estimates     Imputations     =         10
Mean estimation                   Number of obs   =        216
                                  Average RVI     =     0.3882
                                  Largest FMI     =     0.4485
                                  Complete DF     =        215
DF adjustment:   Small sample     DF:     min     =      35.59
                                          avg     =     116.62
Within VCE type:     Analytic             max     =     197.64
{\smallskip}
\HLI{18}{\TOPT}\HLI{48}
                  {\VBAR}       Mean   Std. err.     [95\% conf. interval]
\HLI{18}{\PLUS}\HLI{48}
c.accident@lclass {\VBAR}
               1  {\VBAR}   .7144964   .0369935      .6415438    .7874491
               2  {\VBAR}   .9934973   .0135709      .9659633    1.021031
\HLI{18}{\BOTT}\HLI{48}
Note: Numbers of observations in {\bftt{e(_N)}} vary among imputations.
r; t=2.40 14:46:04
{\smallskip}
. set rmsg off
{\smallskip}
\nullskip
\end{stlog}

The name of the latent class variable (here, \stcmd{lclass})
and the number of imputations are required. The seed is optional,
but of course is strongly recommended for reproducibility of the results,
as the underlying data are randomly simulated.
The multiple imputation version is notably faster.

As one of many diagnostic outputs of MI, the increase in variances / standard errors
due to imputations serves as an indication of how much of a problem
would treating the singly imputed (e.g. modal probability) latent classes would have been.
In the above output, the fraction of missing data (FMI)
is 33\% to 40\%, and the relative variance increase (RVI) is the similar range 
from 39\% to 45\%. This means that the analysis with the deterministic
(modal) imputation of the classes would have had standard errors 
that are about 20\% too small.

\begin{stlog}
. webuse gsem_lca1.dta, clear
(Latent class analysis)
{\smallskip}
. quietly gsem (accident play insurance stock <-), logit lclass(C 2)
{\smallskip}
. predict post_1, class(1) classposterior
{\smallskip}
. gen byte lclass_modal = 2 - (post_1 > 0.50)
{\smallskip}
. mean post_1 lclass_modal
{\smallskip}
Mean estimation                            Number of obs = 216
{\smallskip}
\HLI{13}{\TOPT}\HLI{48}
             {\VBAR}       Mean   Std. err.     [95\% conf. interval]
\HLI{13}{\PLUS}\HLI{48}
      post_1 {\VBAR}   .7207539   .0257112      .6700756    .7714321
lclass_modal {\VBAR}   1.328704   .0320361      1.265559    1.391849
\HLI{13}{\BOTT}\HLI{48}
{\smallskip}
. mean accident, over(lclass_modal)
{\smallskip}
Mean estimation                                       Number of obs = 216
{\smallskip}
\HLI{24}{\TOPT}\HLI{48}
                        {\VBAR}       Mean   Std. err.     [95\% conf. interval]
\HLI{24}{\PLUS}\HLI{48}
c.accident@lclass_modal {\VBAR}
                     1  {\VBAR}   .6896552   .0385529      .6136651    .7656452
                     2  {\VBAR}          1          0             .           .
\HLI{24}{\BOTT}\HLI{48}
\nullskip
\end{stlog}


\subsection{NHANES complex survey data example}

In many important and realistic applications of LCA, including the case
that necessitated the development of this package, the data come from
complex survey designs that require setting the data up for the appropriate
survey-design adjusted analyses. See \svyref{svyset}, \miref{mi svyset},
and \citet{kolenikov:pitblado:2014}.

The standard data set for the \svyref{} commands is an extract from 
the National Health and Nutrition Examination Survey, Round Two
(NHANES II) data. I will use a handful of binary health outcomes
and one ordinal outcome to demonstrate LCA; the ordinal outcome
is arguably an extension that is not quite well covered in the
``classical'' social science LCA.

\begin{stlog}
. frame change default
{\smallskip}
. cap frame nhanes2: clear
{\smallskip}
. cap frame drop nhanes2
{\smallskip}
. frame create nhanes2
{\smallskip}
. frame change nhanes2
{\smallskip}
. 
. webuse nhanes2.dta, clear
{\smallskip}
. svyset
{\smallskip}
Sampling weights: finalwgt
             VCE: linearized
     Single unit: missing
        Strata 1: strata
 Sampling unit 1: psu
           FPC 1: <zero>
{\smallskip}
. svy , subpop(if hlthstat<8) : ///
>         gsem ///
>                 (heartatk diabetes highbp <-, logit) ///
>                 (hlthstat <-, ologit) /// 
>         , lclass(C 2) nolog  startvalues(randomid, draws(5) seed(101)) 
(running {\bftt{gsem}} on estimation sample)
{\smallskip}
Survey: Generalized structural equation model
{\smallskip}
Number of strata = 31                            Number of obs   =      10,351
Number of PSUs   = 62                            Population size = 117,157,513
                                                 Subpop. no. obs =      10,335
                                                 Subpop. size    = 116,997,257
                                                 Design df       =          31
{\smallskip}
\HLI{13}{\TOPT}\HLI{64}
             {\VBAR}             Linearized
             {\VBAR} Coefficient  std. err.      t    P>|t|     [95\% conf. interval]
\HLI{13}{\PLUS}\HLI{64}
1.C          {\VBAR}  (base outcome)
\HLI{13}{\PLUS}\HLI{64}
2.C          {\VBAR}
       _cons {\VBAR}   1.330043   .1259401    10.56   0.000     1.073186    1.586899
\HLI{13}{\BOTT}\HLI{64}
{\smallskip}
Class:    1        
{\smallskip}
Response: heartatk                                      Number of obs = 10,335
Family:   Bernoulli
Link:     Logit    
{\smallskip}
Response: diabetes                                      Number of obs = 10,335
Family:   Bernoulli
Link:     Logit    
{\smallskip}
Response: highbp                                        Number of obs = 10,335
Family:   Bernoulli
Link:     Logit    
{\smallskip}
Response: hlthstat                                      Number of obs = 10,335
Family:   Ordinal  
Link:     Logit    
{\smallskip}
\HLI{13}{\TOPT}\HLI{64}
             {\VBAR}             Linearized
             {\VBAR} Coefficient  std. err.      t    P>|t|     [95\% conf. interval]
\HLI{13}{\PLUS}\HLI{64}
heartatk     {\VBAR}
       _cons {\VBAR}  -1.874967   .1150791   -16.29   0.000    -2.109672   -1.640261
\HLI{13}{\PLUS}\HLI{64}
diabetes     {\VBAR}
       _cons {\VBAR}  -1.785271   .0805057   -22.18   0.000    -1.949463   -1.621078
\HLI{13}{\PLUS}\HLI{64}
highbp       {\VBAR}
       _cons {\VBAR}   .4244921    .076861     5.52   0.000     .2677332    .5812511
\HLI{13}{\PLUS}\HLI{64}
/hlthstat    {\VBAR}
        cut1 {\VBAR}  -3.659014   .8903346                     -5.474863   -1.843165
        cut2 {\VBAR}  -2.272516   .4402984                      -3.17051   -1.374521
        cut3 {\VBAR}  -.2566588   .2032721                      -.671235    .1579173
        cut4 {\VBAR}   1.229244   .1951641                      .8312038    1.627283
\HLI{13}{\BOTT}\HLI{64}
{\smallskip}
Class:    2        
{\smallskip}
Response: heartatk                                      Number of obs = 10,335
Family:   Bernoulli
Link:     Logit    
{\smallskip}
Response: diabetes                                      Number of obs = 10,335
Family:   Bernoulli
Link:     Logit    
{\smallskip}
Response: highbp                                        Number of obs = 10,335
Family:   Bernoulli
Link:     Logit    
{\smallskip}
Response: hlthstat                                      Number of obs = 10,335
Family:   Ordinal  
Link:     Logit    
{\smallskip}
\HLI{13}{\TOPT}\HLI{64}
             {\VBAR}             Linearized
             {\VBAR} Coefficient  std. err.      t    P>|t|     [95\% conf. interval]
\HLI{13}{\PLUS}\HLI{64}
heartatk     {\VBAR}
       _cons {\VBAR}  -6.081307   .6280801    -9.68   0.000    -7.362285   -4.800329
\HLI{13}{\PLUS}\HLI{64}
diabetes     {\VBAR}
       _cons {\VBAR}  -5.223215   .6044468    -8.64   0.000    -6.455993   -3.990438
\HLI{13}{\PLUS}\HLI{64}
highbp       {\VBAR}
       _cons {\VBAR}  -.8166105   .0750027   -10.89   0.000    -.9695795   -.6636415
\HLI{13}{\PLUS}\HLI{64}
/hlthstat    {\VBAR}
        cut1 {\VBAR}   -.657824   .0483113                     -.7563555   -.5592926
        cut2 {\VBAR}   .7123144   .0649814                      .5797839    .8448448
        cut3 {\VBAR}   2.647239   .1192958                      2.403934    2.890544
        cut4 {\VBAR}   24.64389    14.1421                     -4.199113    53.48689
\HLI{13}{\BOTT}\HLI{64}
{\smallskip}
. set rmsg on     
r; t=0.00 14:46:16
{\smallskip}
. estat lcprob
{\smallskip}
Latent class marginal probabilities
{\smallskip}
Number of strata = 31                            Number of obs   =      10,351
Number of PSUs   = 62                            Population size = 117,157,513
                                                 Design df       =          31
{\smallskip}
\HLI{13}{\TOPT}\HLI{48}
             {\VBAR}            Delta-method
             {\VBAR}     Margin   std. err.     [95\% conf. interval]
\HLI{13}{\PLUS}\HLI{48}
           C {\VBAR}
          1  {\VBAR}   .2091523   .0208315      .1698206    .2547976
          2  {\VBAR}   .7908477   .0208315      .7452024    .8301794
\HLI{13}{\BOTT}\HLI{48}
r; t=10.03 14:46:26
{\smallskip}
. estat lcmean
{\smallskip}
Latent class marginal means
{\smallskip}
Number of strata = 31                            Number of obs   =      10,351
Number of PSUs   = 62                            Population size = 117,157,513
                                                 Design df       =          31
{\smallskip}
\HLI{13}{\TOPT}\HLI{48}
             {\VBAR}            Delta-method
             {\VBAR}     Margin   std. err.     [95\% conf. interval]
\HLI{13}{\PLUS}\HLI{48}
1            {\VBAR}
    heartatk {\VBAR}   .1329681   .0132672      .1081603    .1624295
    diabetes {\VBAR}   .1436535   .0099036      .1246119    .1650562
      highbp {\VBAR}   .6045577    .018375      .5665363    .6413552
             {\VBAR}
    hlthstat {\VBAR}
  Excellent  {\VBAR}   .0251111   .0217959      .0041733    .1366775
  Very good  {\VBAR}   .0683138    .021455      .0355579     .127263
       Good  {\VBAR}   .3427603   .0254834      .2928195    .3964437
       Fair  {\VBAR}   .3375009   .0210993      .2958981    .3817814
       Poor  {\VBAR}   .2263139   .0341724      .1642029    .3033906
\HLI{13}{\PLUS}\HLI{48}
2            {\VBAR}
    heartatk {\VBAR}     .00228   .0014287      .0006343    .0081599
    diabetes {\VBAR}   .0053611   .0032231      .0015686    .0181559
      highbp {\VBAR}   .3064836   .0159419      .2749643    .3399221
             {\VBAR}
    hlthstat {\VBAR}
  Excellent  {\VBAR}   .3412286     .01086       .319438    .3637112
  Very good  {\VBAR}   .3296838   .0082507      .3130796     .346724
       Good  {\VBAR}   .2629283   .0094265      .2441597    .2826002
       Fair  {\VBAR}   .0661594   .0073704       .052623    .0828732
       Poor  {\VBAR}   1.98e-11   2.81e-10      5.68e-24    .9857545
\HLI{13}{\BOTT}\HLI{48}
r; t=80.58 14:47:47
{\smallskip}
. set rmsg off    
{\smallskip}
\nullskip
\end{stlog}

This analysis approximates breaking down the population
into "generally healthy" and "unhealthy" groups, as e.g.
the gradient of \textit{hlthstat} variable between the classes
shows. The official \stcmd{gsem postestimation} commands
take approximately forever to run (there is underlying
\stcmd{margins} implementation with iterations over 
the numeric derivatives step size used to compute the stadnard errors).
There is an interaction of \stcmd{svy} and \stcmd{gsem} in that
\stcmd{svy} forces its own starting values that happen to be 
infeasible for LCA, hence the need to specify the initial random search.
The use of the \stcmd{postlca\_class\_predpute} command
makes it possible to run the analylsis much faster,
and to conduct complementary analyses,
e.g. analysis of the racial composition of the two classes.

\begin{stlog}
. set rmsg on
r; t=0.00 14:47:47
{\smallskip}
. postlca_class_predpute, lcimpute(lclass) addm(10) seed(5678)
(10,351 missing values generated)
(10 imputations added; {\sltt{M}} = 10)
{\smallskip}
Sampling weights: finalwgt
             VCE: linearized
     Single unit: missing
        Strata 1: strata
 Sampling unit 1: psu
           FPC 1: <zero>
r; t=0.22 14:47:47
{\smallskip}
. mi estimate : prop lclass
{\smallskip}
Multiple-imputation estimates     Imputations     =         10
Proportion estimation             Number of obs   =     10,351
                                  Average RVI     =     0.5318
                                  Largest FMI     =     0.3641
                                  Complete DF     =      10350
DF adjustment:   Small sample     DF:     min     =      73.85
                                          avg     =      73.85
Within VCE type:     Analytic             max     =      73.85
{\smallskip}
\HLI{13}{\TOPT}\HLI{48}
             {\VBAR}                                   Normal
             {\VBAR} Proportion   Std. err.     [95\% conf. interval]
\HLI{13}{\PLUS}\HLI{48}
      lclass {\VBAR}
          1  {\VBAR}   .2675394   .0053851       .256809    .2782698
          2  {\VBAR}   .7324606   .0053851      .7217302     .743191
\HLI{13}{\BOTT}\HLI{48}
r; t=1.53 14:47:49
{\smallskip}
. mi estimate : prop hlthstat if hlthstat < 8, over(lclass)
{\smallskip}
Multiple-imputation estimates     Imputations     =         10
Proportion estimation             Number of obs   =     10,335
                                  Average RVI     =          .
                                  Largest FMI     =          .
                                  Complete DF     =      10334
DF adjustment:   Small sample     DF:     min     =      36.31
                                          avg     =          .
Within VCE type:     Analytic             max     =          .
{\smallskip}
\HLI{16}{\TOPT}\HLI{48}
                {\VBAR}                                   Normal
                {\VBAR} Proportion   Std. err.     [95\% conf. interval]
\HLI{16}{\PLUS}\HLI{48}
hlthstat@lclass {\VBAR}
   Excellent 1  {\VBAR}   .0196036   .0034636      .0126501    .0265572
   Excellent 2  {\VBAR}   .3104144   .0055292      .2995676    .3212613
   Very good 1  {\VBAR}   .0550625   .0061217      .0426506    .0674743
   Very good 2  {\VBAR}   .3217978   .0056545      .3106989    .3328966
        Good 1  {\VBAR}   .2971359   .0106787      .2758813    .3183905
        Good 2  {\VBAR}   .2795891   .0056354      .2685025    .2906756
        Fair 1  {\VBAR}   .3635286   .0106978      .3423564    .3847008
        Fair 2  {\VBAR}   .0881987    .003944      .0803607    .0960367
        Poor 1  {\VBAR}   .2646694   .0089821      .2470257     .282313
        Poor 2  {\VBAR}          0  (no observations)
\HLI{16}{\BOTT}\HLI{48}
Note: Numbers of observations in {\bftt{e(_N)}} vary among imputations.
r; t=3.01 14:47:52
{\smallskip}
. mi estimate : prop race, over(lclass)
{\smallskip}
Multiple-imputation estimates     Imputations     =         10
Proportion estimation             Number of obs   =     10,351
                                  Average RVI     =     0.4677
                                  Largest FMI     =     0.4671
                                  Complete DF     =      10350
DF adjustment:   Small sample     DF:     min     =      45.27
                                          avg     =     119.43
Within VCE type:     Analytic             max     =     321.80
{\smallskip}
\HLI{13}{\TOPT}\HLI{48}
             {\VBAR}                                   Normal
             {\VBAR} Proportion   Std. err.     [95\% conf. interval]
\HLI{13}{\PLUS}\HLI{48}
 race@lclass {\VBAR}
    White 1  {\VBAR}    .839782   .0093485      .8209563    .8586078
    White 2  {\VBAR}   .8889046   .0042799      .8804196    .8973897
    Black 1  {\VBAR}   .1434614   .0085447      .1263568     .160566
    Black 2  {\VBAR}   .0908358   .0038489      .0832192    .0984524
    Other 1  {\VBAR}   .0167566   .0031195      .0105146    .0229986
    Other 2  {\VBAR}   .0202596   .0017697      .0167778    .0237413
\HLI{13}{\BOTT}\HLI{48}
Note: Numbers of observations in {\bftt{e(_N)}} vary among imputations.
r; t=2.98 14:47:55
{\smallskip}
. set rmsg off
{\smallskip}
\nullskip
\end{stlog}

\subsection{Choosing the number of imputations}

One ``researcher's degrees of freedom'' aspect of this analysis
is the number of imputations $M$ that need to be created.
What this number affects the most is the stability of the standard
errors obtained through the multiple imputation process.
This stability is internally assessed with estimated
degrees of freedom associated with the variance estimate
\citep{barnard:rubin:1999}. With $M=10$ imputations,
the smaller ``poor health'' class have about 50 degrees of freedom:

\begin{stlog}
. mi estimate : prop race, over(lclass) 
{\smallskip}
Multiple-imputation estimates     Imputations     =         10
Proportion estimation             Number of obs   =     10,351
                                  Average RVI     =     0.4677
                                  Largest FMI     =     0.4671
                                  Complete DF     =      10350
DF adjustment:   Small sample     DF:     min     =      45.27
                                          avg     =     119.43
Within VCE type:     Analytic             max     =     321.80
{\smallskip}
\HLI{13}{\TOPT}\HLI{48}
             {\VBAR}                                   Normal
             {\VBAR} Proportion   Std. err.     [95\% conf. interval]
\HLI{13}{\PLUS}\HLI{48}
 race@lclass {\VBAR}
    White 1  {\VBAR}    .839782   .0093485      .8209563    .8586078
    White 2  {\VBAR}   .8889046   .0042799      .8804196    .8973897
    Black 1  {\VBAR}   .1434614   .0085447      .1263568     .160566
    Black 2  {\VBAR}   .0908358   .0038489      .0832192    .0984524
    Other 1  {\VBAR}   .0167566   .0031195      .0105146    .0229986
    Other 2  {\VBAR}   .0202596   .0017697      .0167778    .0237413
\HLI{13}{\BOTT}\HLI{48}
Note: Numbers of observations in {\bftt{e(_N)}} vary among imputations.
{\smallskip}
. mi estimate, dftable
{\smallskip}
Multiple-imputation estimates     Imputations     =         10
Proportion estimation             Number of obs   =     10,351
                                  Average RVI     =     0.4677
                                  Largest FMI     =     0.4671
                                  Complete DF     =      10350
DF adjustment:   Small sample     DF:     min     =      45.27
                                          avg     =     119.43
Within VCE type:     Analytic             max     =     321.80
{\smallskip}
\HLI{13}{\TOPT}\HLI{48}
             {\VBAR}                                          Normal
             {\VBAR} Proportion   Std. err.           df   std. err.
\HLI{13}{\PLUS}\HLI{48}
 race@lclass {\VBAR}
    White 1  {\VBAR}    .839782   .0093485          45.3       34.13
    White 2  {\VBAR}   .8889046   .0042799         106.2       18.59
    Black 1  {\VBAR}   .1434614   .0085447          57.9       28.28
    Black 2  {\VBAR}   .0908358   .0038489         126.3       16.62
    Other 1  {\VBAR}   .0167566   .0031195          59.0       27.89
    Other 2  {\VBAR}   .0202596   .0017697         321.8        9.38
\HLI{13}{\BOTT}\HLI{48}
Note: Numbers of observations in {\bftt{e(_N)}} vary among imputations.
\nullskip
\end{stlog}

The public use version of the NHANES II data uses the approximate
design that has 62 PSU in 31 strata, resulting in 31 design degrees 
of freedom. The imputation degrees of freedom barely exceed that.
Let us push the number of imputations up:

\begin{stlog}
. webuse nhanes2.dta, clear
{\smallskip}
. qui svy , subpop(if hlthstat<8) : ///
>         gsem ///
>                 (heartatk diabetes highbp <-, logit) ///
>                 (hlthstat <-, ologit) /// 
>         , lclass(C 2) nolog  startvalues(randomid, draws(5) seed(101)) 
{\smallskip}
. postlca_class_predpute, lcimpute(lclass) addm(62) seed(9752)
(10,351 missing values generated)
(62 imputations added; {\sltt{M}} = 62)
{\smallskip}
Sampling weights: finalwgt
             VCE: linearized
     Single unit: missing
        Strata 1: strata
 Sampling unit 1: psu
           FPC 1: <zero>
{\smallskip}
. mi estimate : prop race, over(lclass) 
{\smallskip}
Multiple-imputation estimates     Imputations     =         62
Proportion estimation             Number of obs   =     10,351
                                  Average RVI     =     0.2483
                                  Largest FMI     =     0.2893
                                  Complete DF     =      10350
DF adjustment:   Small sample     DF:     min     =     671.70
                                          avg     =   1,786.62
Within VCE type:     Analytic             max     =   3,092.35
{\smallskip}
\HLI{13}{\TOPT}\HLI{48}
             {\VBAR}                                   Normal
             {\VBAR} Proportion   Std. err.     [95\% conf. interval]
\HLI{13}{\PLUS}\HLI{48}
 race@lclass {\VBAR}
    White 1  {\VBAR}   .8369206   .0079313      .8213584    .8524827
    White 2  {\VBAR}   .8900061   .0038498      .8824571     .897555
    Black 1  {\VBAR}   .1455849    .007628      .1306162    .1605536
    Black 2  {\VBAR}   .0900031   .0035541      .0830334    .0969728
    Other 1  {\VBAR}   .0174946   .0029465      .0117091      .02328
    Other 2  {\VBAR}   .0199908    .001709        .01664    .0233417
\HLI{13}{\BOTT}\HLI{48}
Note: Numbers of observations in {\bftt{e(_N)}} vary among imputations.
{\smallskip}
. mi estimate, dftable
{\smallskip}
Multiple-imputation estimates     Imputations     =         62
Proportion estimation             Number of obs   =     10,351
                                  Average RVI     =     0.2483
                                  Largest FMI     =     0.2893
                                  Complete DF     =      10350
DF adjustment:   Small sample     DF:     min     =     671.70
                                          avg     =   1,786.62
Within VCE type:     Analytic             max     =   3,092.35
{\smallskip}
\HLI{13}{\TOPT}\HLI{48}
             {\VBAR}                                          Normal
             {\VBAR} Proportion   Std. err.           df   std. err.
\HLI{13}{\PLUS}\HLI{48}
 race@lclass {\VBAR}
    White 1  {\VBAR}   .8369206   .0079313        1100.3       13.14
    White 2  {\VBAR}   .8900061   .0038498        2639.5        7.08
    Black 1  {\VBAR}   .1455849    .007628        1004.8       13.98
    Black 2  {\VBAR}   .0900031   .0035541        2211.1        8.08
    Other 1  {\VBAR}   .0174946   .0029465         671.7       18.45
    Other 2  {\VBAR}   .0199908    .001709        3092.4        6.26
\HLI{13}{\BOTT}\HLI{48}
Note: Numbers of observations in {\bftt{e(_N)}} vary among imputations.
\nullskip
\end{stlog}

The MI degrees of freedom are now comfortably above 600.
In many i.i.d. data situations, increasing the number of imputations
to several dozens can often send the MI degrees of freedom to 
approximate infinity (the reported numbers are in hundreds 
of thousands). With complex survey designs that have limited
degrees of freedom within each implicate, this may not materialize.
Researchers are encouraged to adopt the workflow where, in parallel,
they

\begin{enumerate}
  \item start with a small number of imputations, like \stcmd{addm(10)}
        in the example above, and develop the analysis code for all
        the substantive analyses, and
  \item working with the key outcomes or analyses, 
        experiment with several values of $M$ to find a reasonable
        trade-off when degrees of freedom exceed the sample size 
        for i.i.d. data, and/or exceed the design degrees of freedom
        for complex survey data by a factor of 3--5.
\end{enumerate}

Then a chosen value of $M$ can be used for all analyses in the paper.

Even a large number of replications may not protect the researcher
from classes that may have structural zeroes. These produce
zero standard errors and missing degrees of freedom and 
variance increase statistics:

\begin{stlog}
{\smallskip}
. 
. mi estimate , dftable : prop hlthstat if hlthstat < 8, over(lclass) 
{\smallskip}
Multiple-imputation estimates     Imputations     =         62
Proportion estimation             Number of obs   =     10,335
                                  Average RVI     =          .
                                  Largest FMI     =          .
                                  Complete DF     =      10334
DF adjustment:   Small sample     DF:     min     =     234.37
                                          avg     =          .
Within VCE type:     Analytic             max     =          .
{\smallskip}
\HLI{16}{\TOPT}\HLI{48}
                {\VBAR}                                          Normal
                {\VBAR} Proportion   Std. err.           df   std. err.
\HLI{16}{\PLUS}\HLI{48}
hlthstat@lclass {\VBAR}
   Excellent 1  {\VBAR}   .0183993   .0033919         309.7       32.68
   Excellent 2  {\VBAR}   .3111818   .0054853        6242.4        3.09
   Very good 1  {\VBAR}   .0573944   .0062471         234.4       41.21
   Very good 2  {\VBAR}   .3212476   .0056361        4001.1        5.03
        Good 1  {\VBAR}   .2947403   .0104573         575.3       20.57
        Good 2  {\VBAR}   .2804536   .0055871        2155.5        8.23
        Fair 1  {\VBAR}   .3656239    .010418        1033.8       13.71
        Fair 2  {\VBAR}    .087117     .00393         549.1       21.27
        Poor 1  {\VBAR}   .2638421   .0087535        4603.6        4.40
        Poor 2  {\VBAR}          0  (no observations)
\HLI{16}{\BOTT}\HLI{48}
Note: Numbers of observations in {\bftt{e(_N)}} vary among imputations.
{\smallskip}
. 
\nullskip
\end{stlog}

\section{Simulation}

So how much of a problem is the modal imputation of the latent classes?
I ran a small simulation to investigate. Samples were taken from
a model with the classes, five binary indicators and one additional 
continuous variable not used in the model as follows:

\begin{tabular}{l|ccc}
   \noalign{\smallskip}
  \hline
  \noalign{\smallskip}
  Variable & Class 1 & Class 2 & Class 3 \\
  \noalign{\smallskip}
  \hline
  \noalign{\smallskip}
  Class probability & 0.4 & 0.4 & 0.2 \\
  $\mathbb{P}[y_1=1|C]$ & 0.7 & 0.3 & 0.6 \\
  $\mathbb{P}[y_2=1|C]$ & 0.8 & 0.5 & 0.6 \\
  $\mathbb{P}[y_3=1|C]$ & 0.5 & 0.4 & 0.7 \\
  $\mathbb{P}[y_4=1|C]$ & 0.5 & 0.3 & 0.7 \\
  $\mathbb{P}[y_5=1|C]$ & 0.8 & 0.4 & 0.3 \\
  \noalign{\smallskip}
  \hline
  \noalign{\smallskip}  
  $y_6\sim N(\mu_c,1)$ & 1 & $\sqrt{2}=1.41$ & $\sqrt{3}=1.73$ \\
  \noalign{\smallskip}
  \hline
  \noalign{\smallskip}  
\end{tabular}

The LCA model with outcomes $y_1$ through $y_5$ was estimated,
and the classes were predicted using the posterior modal prediction,
multiple imputation with $M=5$ imputed data sets, and $M=50$ imputed 
data sets.
(To be precise, there was a single imputation with $M=50$, but 
two versions were run: \stcmd{mi estimate , imp(1/5)} for the limited
application with $M=5$, and without the \stcmd{imp()} option for the full
set of $M=50$.) Inspecting the individual runs visually, a shorter set
of imputations often results in insufficient detail, namely failing to 
capture realizations of (posterior) rare classes.

There were at least two complications with the simulation. 
First, the classes in any LCA model are only identified up to
a permutation of the class labels. 
To wit, there are no distinguishable differences between 
estimates when say classes 1 and 2 are swapped in a model with 2+ classes.
The likelihoods are the same, the point estimates are likely to be
the same within numeric accuracy.
In any particular run of the \stcmd{gsem, lclass()} command, 
the classes depend first and foremost on the starting values. 
The help file 
\stcmd{help gsem\_estimation\_options}\verb|##|\stcmd{startvalues()}
outlines the possible options:

\begin{itemize}
  \item In my own practical work, I typically use 
    \stcmd{startvalues(randomid), draws(10)} or \stcmd{draws(20)}
    to get this many random assignments of the starting classes,
    and having Stata run the EM algorithm to get some decent 
    starting values for the gradient-based optimization methods.
  \item For the simulation purposes, you have to fix either 
    the initial assignment of classes, or the the starting values 
    of the estimates. The former is implementable with 
    \stcmd{startvalues(classid \textit{true\_class})} since 
    the latter is, of course, known. Strangely, in this simulation,
    this did not work out well as it resulted in convergence issues:
    it has been pushing the model into areas of the parameter space
    where the likelihood was too flat to climb out of.
  \item The resulting simulation specification I used was a combination
    of \stcmd{from(b0) startvalues(jitter 0.1, draws(10))}. 
    The value of the starting matrix \stcmd{b0} was obtained
    from a sample of size $10^6$ computed once outside of the simulation.
    Jitter was added to allow some exploration of the sample optima 
    near that point.
\end{itemize}

The second complication of the simulation was that in some runs,
the \stcmd{mi estimate} calls with imputed classes were returning 
errors. This may be a risk of a third-party written imputation routine.
The author will investigate this with Stata Corp developers as this
article progresses.

\begin{stlog}
  . tabulate method
{\smallskip}
      method {\VBAR}      Freq.     Percent        Cum.
\HLI{13}{\PLUS}\HLI{35}
       modal {\VBAR}      2,047       37.29       37.29
  predpute_5 {\VBAR}      1,836       33.44       70.73
 predpute_50 {\VBAR}      1,607       29.27      100.00
\HLI{13}{\PLUS}\HLI{35}
       Total {\VBAR}      5,490      100.00
{\smallskip}
\nullskip
\end{stlog}
  
  

With these limitations in mind, here are the simulation results.

\begin{stlog} 
  \input{imul_class1pr.log.tex}\nullskip
\end{stlog}

\begin{stlog} 
  \input{imul_class1pr_se.log.tex}\nullskip
\end{stlog}

\begin{stlog} 
  . mean y1cl2pr if touse3 \& y1cl2pr>0.01 \& y1cl2pr<0.99, over(n_method)
{\smallskip}
Mean estimation                                Number of obs = 4,503
{\smallskip}
\HLI{19}{\TOPT}\HLI{48}
                   {\VBAR}       Mean   Std. err.     [95\% conf. interval]
\HLI{19}{\PLUS}\HLI{48}
c.y1cl2pr@n_method {\VBAR}
            modal  {\VBAR}    .218858   .0027115      .2135422    .2241739
       predpute_5  {\VBAR}   .2953609   .0020341       .291373    .2993489
      predpute_50  {\VBAR}   .2954626    .002029      .2914848    .2994404
\HLI{19}{\BOTT}\HLI{48}
\nullskip
\end{stlog}

\begin{stlog} 
  . bysort method (rep): sum y1cl2pr if touse3
{\smallskip}
\HLI{158}
-> method = modal
{\smallskip}
    Variable {\VBAR}        Obs        Mean    Std. dev.       Min        Max
\HLI{13}{\PLUS}\HLI{57}
     y1cl2pr {\VBAR}        212    .1846224    .1197415          0   .5820896
{\smallskip}
\HLI{158}
-> method = predpute_5
{\smallskip}
    Variable {\VBAR}        Obs        Mean    Std. dev.       Min        Max
\HLI{13}{\PLUS}\HLI{57}
     y1cl2pr {\VBAR}        210    .2890356    .0949123          0   .4825946
{\smallskip}
\HLI{158}
-> method = predpute_50
{\smallskip}
    Variable {\VBAR}        Obs        Mean    Std. dev.       Min        Max
\HLI{13}{\PLUS}\HLI{57}
     y1cl2pr {\VBAR}        210    .2885514    .0944235          0   .4498129
{\smallskip}
{\smallskip}
. mean y1cl2se if touse3, over(n_method)
{\smallskip}
Mean estimation                                  Number of obs = 632
{\smallskip}
\HLI{19}{\TOPT}\HLI{48}
                   {\VBAR}       Mean   Std. err.     [95\% conf. interval]
\HLI{19}{\PLUS}\HLI{48}
c.y1cl2se@n_method {\VBAR}
            modal  {\VBAR}   .0168442   .0005632      .0157383    .0179501
       predpute_5  {\VBAR}   .0287789   .0006908      .0274223    .0301354
      predpute_50  {\VBAR}   .0282994   .0006203      .0270814    .0295174
\HLI{19}{\BOTT}\HLI{48}
\nullskip
\end{stlog}

\begin{stlog} 
  . mean y4cl1pr if touse3 \& y4cl1pr>0.01 \& y4cl1pr<0.99, over(n_method)
{\smallskip}
Mean estimation                                Number of obs = 4,687
{\smallskip}
\HLI{19}{\TOPT}\HLI{48}
                   {\VBAR}       Mean   Std. err.     [95\% conf. interval]
\HLI{19}{\PLUS}\HLI{48}
c.y4cl1pr@n_method {\VBAR}
            modal  {\VBAR}   .5134519   .0027162      .5081269    .5187769
       predpute_5  {\VBAR}   .4861217   .0018979      .4824009    .4898424
      predpute_50  {\VBAR}   .4862154    .001881      .4825277     .489903
\HLI{19}{\BOTT}\HLI{48}
\nullskip
\end{stlog}

\begin{stlog} 
  . bysort method (rep): sum y4cl1pr if touse3
{\smallskip}
\HLI{158}
-> method = modal
{\smallskip}
    Variable {\VBAR}        Obs        Mean    Std. dev.       Min        Max
\HLI{13}{\PLUS}\HLI{57}
     y4cl1pr {\VBAR}        212    .4693585    .1835638          0          1
{\smallskip}
\HLI{158}
-> method = predpute_5
{\smallskip}
    Variable {\VBAR}        Obs        Mean    Std. dev.       Min        Max
\HLI{13}{\PLUS}\HLI{57}
     y4cl1pr {\VBAR}        210     .478399    .0931995          0   .7121175
{\smallskip}
\HLI{158}
-> method = predpute_50
{\smallskip}
    Variable {\VBAR}        Obs        Mean    Std. dev.       Min        Max
\HLI{13}{\PLUS}\HLI{57}
     y4cl1pr {\VBAR}        210    .4790298    .0918895          0   .7082694
{\smallskip}
{\smallskip}
. mean y4cl1se if touse3, over(n_method)
{\smallskip}
Mean estimation                                  Number of obs = 632
{\smallskip}
\HLI{19}{\TOPT}\HLI{48}
                   {\VBAR}       Mean   Std. err.     [95\% conf. interval]
\HLI{19}{\PLUS}\HLI{48}
c.y4cl1se@n_method {\VBAR}
            modal  {\VBAR}   .0233971   .0006353      .0221495    .0246447
       predpute_5  {\VBAR}   .0343111   .0007482      .0328417    .0357804
      predpute_50  {\VBAR}    .033702   .0006853      .0323564    .0350477
\HLI{19}{\BOTT}\HLI{48}
\nullskip
\end{stlog}

\begin{stlog} 
  . mean y6cl1pr if touse3, over(n_method)
{\smallskip}
Mean estimation                                Number of obs = 4,797
{\smallskip}
\HLI{19}{\TOPT}\HLI{48}
                   {\VBAR}       Mean   Std. err.     [95\% conf. interval]
\HLI{19}{\PLUS}\HLI{48}
c.y6cl1pr@n_method {\VBAR}
            modal  {\VBAR}   1.178873   .0019722      1.175007     1.18274
       predpute_5  {\VBAR}   1.208982   .0015596      1.205925     1.21204
      predpute_50  {\VBAR}   1.209956   .0014737      1.207067    1.212845
\HLI{19}{\BOTT}\HLI{48}
\nullskip
\end{stlog}

\begin{stlog} 
  . bysort method (rep): sum y6cl1pr if touse3
{\smallskip}
\HLI{158}
-> method = modal
{\smallskip}
    Variable {\VBAR}        Obs        Mean    Std. dev.       Min        Max
\HLI{13}{\PLUS}\HLI{57}
     y6cl1pr {\VBAR}        212    1.178084     .076269   .9719405   1.351546
{\smallskip}
\HLI{158}
-> method = predpute_5
{\smallskip}
    Variable {\VBAR}        Obs        Mean    Std. dev.       Min        Max
\HLI{13}{\PLUS}\HLI{57}
     y6cl1pr {\VBAR}        210    1.210396    .0638459   1.050905   1.356989
{\smallskip}
\HLI{158}
-> method = predpute_50
{\smallskip}
    Variable {\VBAR}        Obs        Mean    Std. dev.       Min        Max
\HLI{13}{\PLUS}\HLI{57}
     y6cl1pr {\VBAR}        210    1.212146     .060572   1.057609   1.368742
{\smallskip}
{\smallskip}
. mean y6cl1se if touse3, over(n_method)
{\smallskip}
Mean estimation                                  Number of obs = 632
{\smallskip}
\HLI{19}{\TOPT}\HLI{48}
                   {\VBAR}       Mean   Std. err.     [95\% conf. interval]
\HLI{19}{\PLUS}\HLI{48}
c.y6cl1se@n_method {\VBAR}
            modal  {\VBAR}   .0586438   .0012751      .0561397    .0611478
       predpute_5  {\VBAR}   .0725802   .0016715      .0692977    .0758627
      predpute_50  {\VBAR}   .0712979   .0014423      .0684657    .0741301
\HLI{19}{\BOTT}\HLI{48}
\nullskip
\end{stlog}

\begin{stlog} 
  . mean y6cl3pr if touse3, over(n_method)
{\smallskip}
Mean estimation                                  Number of obs = 621
{\smallskip}
\HLI{19}{\TOPT}\HLI{48}
                   {\VBAR}       Mean   Std. err.     [95\% conf. interval]
\HLI{19}{\PLUS}\HLI{48}
c.y6cl3pr@n_method {\VBAR}
            modal  {\VBAR}   1.450547   .0084174      1.434017    1.467077
       predpute_5  {\VBAR}   1.406871    .006128      1.394837    1.418906
      predpute_50  {\VBAR}   1.406623    .005551      1.395722    1.417525
\HLI{19}{\BOTT}\HLI{48}
\nullskip
\end{stlog}

\begin{stlog} 
  . bysort method (rep): sum y6cl3pr if touse3
{\smallskip}
\HLI{158}
-> method = modal
{\smallskip}
    Variable {\VBAR}        Obs        Mean    Std. dev.       Min        Max
\HLI{13}{\PLUS}\HLI{57}
     y6cl3pr {\VBAR}      1,514    1.453111     .132573   .9581373   2.053384
{\smallskip}
\HLI{158}
-> method = predpute_5
{\smallskip}
    Variable {\VBAR}        Obs        Mean    Std. dev.       Min        Max
\HLI{13}{\PLUS}\HLI{57}
     y6cl3pr {\VBAR}      1,607     1.40193    .0876854   1.136963   1.883443
{\smallskip}
\HLI{158}
-> method = predpute_50
{\smallskip}
    Variable {\VBAR}        Obs        Mean    Std. dev.       Min        Max
\HLI{13}{\PLUS}\HLI{57}
     y6cl3pr {\VBAR}      1,607    1.401776    .0817985   1.139892   1.755444
{\smallskip}
{\smallskip}
. mean y6cl3se if touse3, over(n_method)
{\smallskip}
Mean estimation                                Number of obs = 4,728
{\smallskip}
\HLI{19}{\TOPT}\HLI{48}
                   {\VBAR}       Mean   Std. err.     [95\% conf. interval]
\HLI{19}{\PLUS}\HLI{48}
c.y6cl3se@n_method {\VBAR}
            modal  {\VBAR}   .0923276   .0012236      .0899288    .0947263
       predpute_5  {\VBAR}   .1099187   .0013831      .1072071    .1126303
      predpute_50  {\VBAR}   .1078059   .0012745      .1053072    .1103046
\HLI{19}{\BOTT}\HLI{48}
\nullskip
\end{stlog}

\newpage

%%% to peek at the Stata Journal features
% discussion of the Stata Journal document class.
% \input sj.tex
% discussion of the Stata Press LaTeX package for Stata output.
% \input stata.tex

\bibliographystyle{sj}
\bibliography{sj}

\begin{aboutauthors}
Stas Kolenikov is Principal Statistician at NORC who has been
using Stata and writing Stata programs for about 25 years.
He had worked on economic welfare and inequality, spatiotemporal
environmental statistics, mixture models, missing data,
multiple imputation, structural equations with latent variables,
resampling methods, complex sampling designs, survey weights,
Bayesian mixed models, combining probability and non-probability samples,
latent class analysis, and likely some other stuff, too.
\end{aboutauthors}

\endinput
